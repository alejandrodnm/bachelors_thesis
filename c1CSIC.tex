%CSIC
\chapter{Entorno institucional} 

En este capítulo se presenta la institución, y el grupo de trabajo perteneciente a dicha institución donde se realizó el proyecto. 

\section{Centro de Automática y Robótica}

	Por un acuerdo entre la \gls{UPM} y el \gls{CSIC}, en 2010 se ha creado el \gls{CAR} como una unidad mixta entre parte del Instituto de Automática Industrial del \gls{CSIC} y la División de Ingeniería de Sistemas y Automática de la \gls{UPM}. El \gls{CAR} se concibe como un centro de investigación en Automática y Robótica cuya estrategia se sustenta en dos grandes pilares: por una parte, la existencia de una actividad investigadora de excelencia que sea referente internacional y, por otra, el impulso de la transferencia de resultados y tecnología a la sociedad y al sector productivo.
	
	La principal seña de identidad del \gls{CAR} será el desarrollo de una investigación de calidad con un planteamiento global y una aproximación multidisciplinar, contribuyendo tanto al avance del conocimiento, como a la resolución de problemas concretos planteados desde distintos ámbitos de la sociedad \cite{car}.
	
\subsection{Objetivos del Centro de Automática y Robótica}

Adquirir conocimientos científicos y tecnológicos al más alto nivel, en el campo de la automatización, cultivando líneas de investigación científica acordes con las prioridades marcadas por el Programa Marco de la Unión Europea y el Plan Nacional de Investigación. Transferir dicha capacidad a la sociedad mediante la implicación del centro en proyectos de innovación. Es decir, participar, realizando tareas de investigación científica y tecnológica, en colaboración con otros organismos y/o empresas, para resolver todos los problemas de desarrollo, industrialización y comercialización que toda auténtica innovación comporta.

\subsection{Grupo AUTOPIA}

El Grupo AUTOPIA se desarrolla en España en el desaparecido Instituto de Automática Industrial, actualmente en el \gls{CAR}, de la \gls{UPM} y el \gls{CSIC}. La línea de investigación principal se orienta hacia la conducción automática de vehículos \cite{utopia}. 

Desde sus inicios en 1998, ha centrado su trabajo en  la aplicación de técnicas de control, desarrolladas primero para robots móviles, en vehículos autónomos reales. Estas técnicas se basan principalemente en lógica difusa, dado que permite el uso de reglas relativamente sencillas para emular el comportamiento humano en conducción de vehículos (\textit{Si el vehículo está desviado hacia la derecha mueve el volante hacia la izquierda}).
El objetivo final es lograr una conducción completamente autónoma, así como mejorar la seguridad en la conducción, principalmente en entornos urbanos y frente a situaciones de alto riesgo. El Grupo ha contado con la financiación proveniente de diversos proyectos de investigación, en la sección \ref{ape:csic} del apéndice, se presentan algunos de ellos.
