\setcounter{page}{4}
\chapter*{Resumen}



Se presenta un sistema de aprendizaje de controladores difusos, orientado a la conducción autónoma de vehículos, particularmente el control de los pedales, el cual utiliza dos técnicas de aprendizaje, una local y una global, para aprender de los valores de entrada y los resultados generados, con el objetivo de modificar la estructura y parámetros de el controlador difuso.

El aprendizaje local se realiza en tiempo real permitiendo afinar los consecuentes de las reglas correspondientes al controlador difuso. Podemos considerarlo como una respuesta a corto plazo, ya que en cada instante de tiempo, analiza el funcionamiento del controlador, para determinar si es necesario ajustar el modelo que va a ser utilizado en el instante siguiente.

La modificación de la topología y la inserción de nuevas funciones de pertenencia, corresponden al aprendizaje global. Al transcurrir un periodo especifico de tiempo, el sistema analiza el historial de valores de entradas utilizados por el controlador y las funciones de pertenencia a las cuales corresponden, y según el criterio establecido, determina si es necesario, modificar la estructura de las funciones de pertenencia o insertar nuevas funciones de pertenencia.

El sistema fue probado en un entorno virtual utilizando unos 30 vehículos, caracterizados por poseer dinámicas diferentes. Una vez determinado el correcto funcionamiento utilizando vehículos virtuales , se procedió a implementar el sistema en un vehículo real con capacidades de conducción automática, en el cual se utilizaron diferentes configuraciones, con la finalidad de realizar un análisis comparativo y determinar cual genera el mejor desempeño. En ambos casos se obtuvieron muy buenos resultados, incluso en comparación con el modo de conducir de una persona.

\newpage

