\newacronym{CSIC}{CSIC}{Consejo Superior de Investigaciones Científicas}
\newacronym{CAR}{CAR}{Centro de Automática y Robótica}
\newacronym{ITS}{ITS}{Sistemas Inteligentes de Transporte (\textit{Intelligent Transportation Systems})}
\newacronym{IAI}{IAI}{Instituto de Automática Industrial}
\newacronym{UPM}{UPM}{Universidad Politécnica de Madrid}
\newacronym{ORBEX}{ORBEX}{Ordenador Borroso Experimental}
\newacronym{ZOCO}{ZOCO}{Zona de Conducción}
\newacronym{DGPS}{DGPS}{Differential Global Positioning System}
\newacronym{GPS}{GPS}{Global Positioning System}
\newacronym{CAN}{CAN}{Control Area Network}
\newacronym{WLAN}{WLAN}{Wireless Area Network}
\newacronym{ABS}{ABS}{Antilock Breaking System}
\newacronym{IMU}{IMU}{Inertial Measurement Unit}
\newacronym{TORCS}{TORCS}{The Open Racing Car Simulator} 
\newacronym{MAE}{MAE}{ error absoluto medio (\textit{Mean Absolute Error})} 
\newacronym{GPL}{GPL}{Licencia Pública General (\textit{General Public License}}
\newglossaryentry{error}
{
  name={\ensuremath{\varepsilon_v(i)}},
  description={Error de velocidad en el instante de tiempo $i$},
  sort=1error
}
\newglossaryentry{Cte}
{
  name={\ensuremath{Cte}},
  description={Constante de normalización},
  sort=constante
}
\newglossaryentry{mu_{i}(t-1)}
{
  name={\ensuremath{mu_{i}(t-1)}},
  description={Grado de activación del singleton $i$ en el instante de tiempo (t-1)},
  sort=1grado
}
\newglossaryentry{s(t)}
{
  name={\ensuremath{s(t)}},
  description={Singleton en el instante $t$},
  sort=singleton
}
\newglossaryentry{P(i)}
{
  name={\ensuremath{P(i)}},
  description={Perfil de velocidad durante el instante de tiempo $i$},
  sort=perfil
}
\newglossaryentry{V(i)}
{
  name={\ensuremath{V(i)}},
  description={Velocidad del velocidad durante el instante de tiempo $i$},
  sort=velocidad
}
\newglossaryentry{a(i)}
{
  name={\ensuremath{a(i)}},
  description={Aceleración del velocidad durante el instante de tiempo $i$},
  sort=aceleracion
}
\newglossaryentry{dedt}
{
  name={\ensuremath{\frac{d(\varepsilon_v(i))}{d(t)}}},
  description={Derivada del error de velocidad con respecto al tiempo},
  sort=1derivada
}
\newglossaryentry{a_c}
{
  name={\ensuremath{a_c}},
  description={Aceleración de confort},
  sort=aceleracionc
}
\newglossaryentry{t(i)}
{
  name={\ensuremath{t(i)}},
  description={tiempo en el instante $i$},
  sort=t
}
\newglossaryentry{summod}
{
  name={\ensuremath{\sum_{x= i-10}^{i}\mid\varepsilon_v(x)\mid}},
  description={Suma del modulo del error de velocidad de los últimos diez instantes de tiempo},
  sort=1sumatoria
}

\newglossaryentry{integrall}
{
  name={\ensuremath{\int_{i-10}^{i}\varepsilon_v(i)}},
  description={Integral del error de velocidad durante los últimos diez instantes de tiempo},
  sort=1integral
}
\newglossaryentry{Microsoft Visual C++ Express}
{	
	name={Microsoft Visual C++ Express},
    description={Entorno de desarrollo integrado (IDE) gratuito desarrollado por Microsoft para lenguajes de programación C, C++ y C++/CLI} 
}
\newglossaryentry{Microsoft Windows XP}
{	
	name={Microsoft Windows XP},
    description={Versión del sistema operativo Microsoft Windows} 
}
\newglossaryentry{C++}
{
	name={C++},
    description={Lenguaje de programación diseñado con la intención de extender al lenguaje de programación C con mecanismos que permitan la manipulación de objetos}   
}
\newglossaryentry{singletons}
{
	name={Singletons},
    description={Conjunto con un único elemento}   
}
\newglossaryentry{Open Source}
{
	name={Open Source},
    description={Código abierto, es el término con el que se conoce al software distribuido y desarrollado libremente}
}

\newglossaryentry{Linux}
{
	name={Linux},
    description={Núcleo libre de sistema operativo basado en Unix}
}

\newglossaryentry{x86}
{
	name={x86},
    description={Nombre dado al grupo de microprocesadores de la familia de Intel y a la arquitectura que comparten estos procesadores}
}

\newglossaryentry{AMD64}
{
	name={AMD64},
    description={Arquitectura basada en la extensión del conjunto de instrucciones x86 para manejar direcciones de 64 bits}
}

\newglossaryentry{PPC}
{	
	name={PPC},
    description={Sistema operativo Linux que posee un kernel nativo PPC}
}

\newglossaryentry{FreeBSD}
{	
	name={FreeBSD},
    description={Sistema operativo libre basado en las CPU de arquitectura Intel}
}

\newglossaryentry{MacOSX}
{
	name={MacOSX},
    description={Sistema operativo desarrollado y comercializado por Apple Inc.}
}

\newglossaryentry{Windows}
{
	name={Windows},
    description={familia de sistemas operativos desarrollados por Microsoft}
}