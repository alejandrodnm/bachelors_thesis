\chapter{CSIC}
\label{ape:csic}


El Grupo ha contado con la financiación proveniente de diversos proyectos de investigación, en la sección \ref{ape:csic}, entre los que cabe destacar:

\begin{itemize}
	
	\item \textbf{\gls{ORBEX}}. Financiado por la CICYT, se definió e implementó el núcleo de un sistema de inferencia difusa, el cual se emplea en la actualidad para dar soporte a los diferentes sistemas de navegación de los vehículos. La estructura del sistema se definirá en detalle en la sección \ref{sec:orbex}.
	
	\item \textbf{\gls{ZOCO}}. Igualmente financiado por la CICYT y que permitió la construcción de la pista de experimentación en la cual se llevan a cabo las pruebas del presente trabajo. También se detallarán los aspectos de la pista de pruebas en la sección \ref{sec:zoco}.
	
	\item \textbf{COVAN} y \textbf{GLOBO}, financiados respectivamente por la Comunidad Autónoma de Madrid y la CICYT, sirvieron para comprar e instrumentar dos fugonetas \textit{Citroën Berlingo} eléctricas con las que se llevaron a cabo los primeros experimentos de conducción autónoma \cite{Alcalde2000}.
	
	\item \textbf{ISAAC} e \textbf{ISAAC-2}, financiados ambos por el extinto Ministerio de Ciencia y Tecnología, donde en colaboración con grupos de la Universidad de Alcalá de Henares, la Universidad Politécnica de Madrid y la Universidad de Extremadura, se desarrollaron sistemas para reconocer el entorno de los vehículos mediante diferentes técnicas.
	
	\item \textbf{CYBERCARS-2} fue un proyecto financiado por la Comunidad Europea, trataba sobre la realización de maniobras cooperativas entre vehículos de diferente naturaleza y en él participaron hasta once instituciones del sector del transporte.
	
	\item \textbf{MARTA} y \textbf{GUIADE} fueron financiados por el Ministerio de Fomento y están en desarrollo; buscan encontrar aplicaciones reales en automoción junto a grandes empresas españolas del sector.
	
	\item \textbf{CITYELEC} es un proyecto singular concedido a finales de 2009 y que, en una línea solidaria con el medio ambiente, busca la implementación en ciudades españolas de un flujo de tráfico verde.
	
\end{itemize}